% Things to do before submission:
% Sally's notes
% Matt's notes
% Remove code to copy refs
% Github DOI?
% Finish abstract
% Finish citations (add citations only mentioned online to MS)

% Notes from call with Rob
% van Tienderen - sensitivity approach better for evolution
% send list of species to rob

% TASKS
% what's going on with monocots?
% in methods - mention stuff about Salisbury
% leaf size

\documentclass[12pt, oneside]{article}

\usepackage{geometry}
\geometry{letterpaper}
\usepackage[parfill]{parskip}
\usepackage{graphicx}
\usepackage{booktabs}
\usepackage{topcapt}
% \usepackage[labelformat=empty]{caption} % for empty caption labels
\usepackage{caption}
\usepackage{amssymb}
\usepackage{amsmath}
\usepackage{natbib}
\usepackage{color}
\usepackage{array}
\usepackage{gensymb}
\usepackage{setspace}
\newcommand{\stretchy}{2}
\usepackage{lineno}
\usepackage{array}
\usepackage{lscape}
\usepackage{multirow}
% \usepackage[none]{hyphenat}

% Make helvetica the default sans-serif font
\renewcommand\sfdefault{phv}

% Package command for citing R packages
\newcommand{\pkg}[1]{{\fontseries{b}\selectfont #1}} 

% Command for abbreviating Ellenberg light indicator value
\newcommand{\el}{L-value} 

\usepackage{Sweave}
\begin{document}
\Sconcordance{concordance:ms.tex:ms.Rnw:%
1 32 1 1 0 34 1 1 10 410 1}


\title{Light and life form interact to shape stomatal ratio among British angiosperms$^1$}
\author{Christopher D. Muir$^2$}
\date{} % delete this line to display the current date

\maketitle

$^1$ Manuscript received \underline{     }; revision accepted \underline{     }. \\

$^2$ Biodiversity Research Centre and Botany Department, University of British Columbia, Vancouver, British Columbia V6T 1Z4, Canada \\
\\
\textit{Author for correspondence:} \\
\textit{Christopher D. Muir} \\
\textit{Tel:} +17782284851 \\
\textit{Email:} chrisdmuir@gmail.com \\
University of British Columbia \\
6270 University Blvd. \\
Vancouver, BC, Canada \\
V6T 1Z4 \\
\\
Short title: Shedding light on stomatal evolution\\
\\
Word count: \\
Summary:  \\
Introduction:  \\
Methods and Results:  \\
Discussion:  \\
\# Figures and \# Tables, \# references

\begin{Schunk}
\begin{Soutput}
[1] TRUE
\end{Soutput}
\end{Schunk}

%--------------------------------------------------
% Acknowledgements and Contributions
%--------------------------------------------------

\section*{Acknowledgements}
I thank Sally Otto, Matt Pennell, and Rob Salguero-G\'{o}mez for feedback on this manuscript.

% \section*{Author contribution statement}
% CDM designed the study, collected data, analyzed the data, and wrote the manuscript.

\linenumbers
\setstretch{\stretchy}
% \pagenumbering{gobble}


\section*{Abstract}

% (1) the research conducted, including the rationale, (2) methods, (3) key results and (4) the main conclusion, including key points of discussion?

\begin{itemize}
	\item In most plants, stomata are located only on the abaxial leaf surface (hypostomy), but many plants have stomata on both surfaces (amphistomy). Variation in stomatal ratio (the ratio of ab- and adaxial stomatal densities) is probably adaptive, but the ecological conditions that favor amphistomy are not well understood. In particular, high light and herbaceous growth form have been hypothesized to favor amphistomy, but these hypotheses have not been rigourously tested together.
	\item I leveraged a large dataset including stomatal ratio, Ellenberg light indicator value, Raunki\ae r lifeform, and phylogenetic relationships for 372 species of British angiosperms. I used phylogenetic comparative methods to test how light and/or growth form influence stomatal ratio.
	\item (return to this) key results: \el, growth form, and interaction are important
	\item I show for the first time that light and growth form interact to shape variation in stomatal ratio; amphistomy is advantageous in high light, but mostly for herbs. These results improve our understanding of the adaptive significance of stomatal ratio, use stomatal ratio as proxy for paleo vegetation, and as a target for crop improvement.
\end{itemize}

% POSSIBLE CONCLUSION: THIS TELLS US THAT CO2 LIMITS PHOTOSYNTHESIS AND HIGH LIGHT, BUT ONLY STRONG SELECTION IN FAST LIFE HISTORY (HERBS)...OR SOMETHING

\section*{Keywords}

Adaptation, amphistomy, Ellenberg light indicator value, growth form, phylogenetic comparative methods, stomata, stomatal ratio

\section*{INTRODUCTION}

%%%%% PARAGRAPH 1-2: general stuff about adaptation, stomata, stomatal ratio %%%%%
Natural selection shapes leaf anatomy in order to optimize its photosynthetic function in a given environment \citep{Haberlandt_1914, Givnish_1987, Smith_etal_1997}. By understanding the adaptive significance of leaf anatomical variation we can learn about natural history, find targets for crop improvement, and identify anatomical proxies for paleoclimates preserved in the fossil record [CITE]. The size, density, and distribution of stomata on a leaf vary widely and impact functions like the maximum photosynthetic rate, water-use efficiency, photosynthetic nitrogen-use efficiency, and susceptibility to foliar pathogens that infect through stomata [CITATIONS]. Hence, stomata have been especially useful in understanding plastic and evolutionary response to climate change and domestication (Royer, Ward, Woodward, Beerling, Milla et al...).

While the density and size of stomata have been researched extensively [CITATIONS], the adaptive significance of stomatal distribution is less well understood. Stomata are most often found only on the lower leaf surface (hypostomy) but occur on both surfaces (amphistomy) in many species \citep{Metcalfe_Chalk_1950, Parkhurst_1978, Mott_etal_1982}. Theory and experiments demonstrate that amphistomy increases photosynthetic rates under many conditions. By creating a second parallel pathway for CO$_2$ diffusion within the mesophyll, amphistomy optimally supplies CO$_2$ \citep{Parkhurst_1978, Gutschick_1984b, Jones_1985}. Amphistomy is correlated with greater CO$_2$ diffusion \citep{Beerling_Kelly_1996} and higher photosynthetic rates \citep{Mckown_etal_2014}. These observations are corroborated by experiments demonstrating that amphistomy increases maximum photosynthetic rates by up to 20\% \citep{Parkhurst_Mott_1990}. On the other hand, amphistomy can increase transpiration \citep{Jones_1985, Foster_Smith_1986, Buckley_etal_2015}. While transition to amphistomy is thus thought to increase transpiration, empirical studies suggest greater water-use efficiency in amphistomatous species \citep{Bucher_etal_2017}. Hence, amphistomy appears to benefit a plant's carbon use relative to water loss and should be favored when CO$_2$ limits photosynthetic rate. The open questions are under what ecological conditions does CO$_2$ supply most strongly limit photosynthetic rate \citep{Peat_Fitter_1994b} and when is photosynthetic rate most important to fitness?

% NOTES FOR FIRST TWO PARAGRAPHS
% Stomatal ratio and photo/water use: Parkhurst papers, 
% from Peat and Fitter 1994b "Jones (1985) found that amphistomaty may be advantageous as it allows greater CO2 transfer through the mesophyll, especially in thicker leaves, and greater leaf conductance in high light conditions when CO2 is limiting photosynthetic rate. However, his models showed that hypostomaty would be advantageous when there are high boundary layer resistances (for example in large leaves or deep within a plant canopy), high humidity or a temperature gradient across the leaf."

%%%%% PARAGRAPH 3-4: adaptive hypotheses and current limitations %%%%%
The leading, nonmutually exclusive hypotheses are that 1) open habitats favour amphistomy because CO$_2$ diffusion most strongly limits photosynthetic rate under high light and 2) herbaceous growth form favours amphistomy because traits that maximize photosynthetic rate are often under stronger selection in herbs. \citeauthor{Salisbury_1927} (\citeyear{Salisbury_1927}) first noted that amphistomy is most common in herbaceous plants from open habitats (i.e., with high light) of the British flora. These observations have been replicated in other studies \citep{Mott_etal_1982, Peat_Fitter_1994b, Jordan_etal_2014, Muir_2015} and may support physiological and ecological hypotheses that CO$_2$ most strongly limits photosynthesis in high light and/or photosynthesis contributes most to fitness in herbaceous plants. Under high light, CO$_2$ can strongly limit maximum photosynthetic rates, espcecially in thick leaves \citep{Jones_1985}. Hence, having stomata on both surfaces relieves this limitation by adding a second parallel pathway for CO$_2$ diffusion. Parkhurst \citeyear{Parkhurst_1978} argued that greater leaf thickness \textit{per se} selected for amphistomy, but there is little evidence for correlations between leaf thickness and stomatal ratio independent of light \citep{Mott_etal_1982, Gibson_1996, Muir_2015}. Amphistomy is correlated with open habitat in warm desert plants of western North America \citep{Mott_etal_1982, Gibson_1996}, among the Proteaceae \citep{Jordan_etal_2014}, and in continental European herbs \citep{Bucher_etal_2017}.

Stomatal ratio is also associated with growth form. In the British flora, \citeauthor{Salisbury_1927} (\citeyear{Salisbury_1927}) found that trees and shurbs are nearly always hypostomatous, whereas herbs from open habitats are amphistomatous. This pattern holds when data are averaged by family to coarsely control for phylogenetic nonindependence \citep{Peat_Fitter_1994b} or when using alternative classification schemes, such as Raunki\ae r life form \citep{Peat_Fitter_1994b}. Across plants from ~90 familes worldwide, growth form is the strongest predictor of stomatal ratio when multiple factors are estimated simultaneously and controlling for phylogenetic nonindependence \citep{Muir_2015}. These patterns are consistent with other data indicating that many herbaceous plants are under strong selection for high maximum photosynthetic rates. \citep{Bazzaz_1979, Korner_etal_1989}. NEED MORE RECENT CITATIONS ON THIS.

% NOTES FOR PARAGRAPHS 3-4:
% From Mott et al 1982: ''Korner, Scheel \& Bauer (1979) have recently reviewed the literature on maximum stomatal conductance and photosynthetic rates for plants of specific habitats and growth forms. Low maximum photosynthetic rates are observed for succulents, evergreen conifers, deciduous woody plants, herbs from shady habitats, evergreen woody plants, and desert and steppe shrubs. Higher maximum values are obtained for deciduous fruit trees, wild grasses, cultivated C$_4$ and C$_3$ grasses, herbaceous crop plants, herbs from open habitats, and plants from aquatic habitats and swamps. Bazzaz (1979) cites evidence that the maximum photosynthetic rate declines through successional stages in a plant community, pointing to higher rates for open field plants and grasses and lower rates for canopy deciduous trees and shade habitat herbs. Early successional trees appear to be intermediate in photosynthetic rates.''

%%%% PARAGRAPH 5-6 %%%%%
Although previous comparative studies have tested whether open habitat and growth form influence stomatal ratio, we do not know if these effects are independent of one another. Open habitat and growth form may not be independent because open habitats generally consist of more short-statured, herbaceous plants. Some authors have attempted to disentangle light and growth form by contrasting herbs from open and understory habiats \citep{Salisbury_1927}. However, this is problematic if phylogenetic relationships are not controlled for, because shade species may share traits simply because they are more closely related to each other than they are to high light species. Finally, open habitat and groth form may also interact with one another. For example, amphistomy may only be favored when CO$_2$ strongly limits photosynthetic rate (e.g. in high light) \textit{and} photosynthetic rate strongly limits fitness (e.g. in herbs).

To better understand the adapative significance of stomatal ratio, I asked three main questions:

\begin{enumerate}

  \item{Are light habitat and growth form correlated?}
  \item{Do light habitat and growth form influence stomatal ratio additively, or do their effects interact?}
  \item{Is evolution of stomatal ratio mediated by changes in stomatal density on the adaxial (upper) surface, abaxial (lower) surface, or both?}
  
\end{enumerate}

The final question is important for telling whether amphistomy is part of a coordinated syndrome of traits that promote higher photosynthetic rate, as both the light and growth form hypotheses assume. If evolved increases in stomatal ratio are mediated by shifting abaxial stomata to the adaxial surface, holding total stomatal density constant, then the overall increase in CO$_2$ diffusion would be limited. In contrast, if amphistomy evolves by increasing adaxial stomatal density while holding abaxial density constant, then \textit{total} stomatal density must increase as well. Evolutionary coordination of amphistomy and high stomatal density would reinforce one another, increasing CO$_2$ supply to chloroplasts more than changes in either trait would in isolation.

To address these questions, I reanalyzed existing data on stomatal ratio, light habitat, and growth form in British angiosperms \citep{Salisbury_1927, Fitter_Peat_1994a, BEF} using phylogenetic comparative methods. The British angiosperm flora is well suited for these questions because this flora has been comprehensively surveyed for many ecologically important traits, meaning it is probably the least biased survey of stomatal trait variation. Salisbury's observations on stomata and ecology in the British flora have heavily influenced plant ecophysiology, but many of his and subsequent authors' analyses have significant limitations because of inadequate statistical methods. For example, few analyses until recently account for phylogenetic nonindependence \citep{Felsenstein_1985}, which can strongly influence inferences on stomatal traits and growth form \citep[this study did not consider light]{Kelly_Beerling_1995}. A species-level phylogeny of the entire British flora \citep{Lim_etal_2014} now allows for the first time rigorous analysis of evolutionary relationships among stomatal ratio, light, and growth form. 

% NOTES , for example, two herbaceous species adapted to different light conditions may not really be comparable since light has a strong impact on life history selection \citep{Franco_Silvertown_1996, Galloway_Etterson_2007}.

% from previous paper
% Here, we examine whether stomata on the abaxial (`lower') surface of the leaf evolve independently of adaxial (`upper') stomata. Stomata are microscopic pores on the leaf surface formed by a pair of guard cells. The density, size, and arrangement of stomata on a leaf set the maximum stomatal conductance to CO$_2$ diffusing into a leaf and the amount of water that transpires from it \citep{Parkhurst_1978, Sack_etal_2003, Franks_Farquhar_2001, Galmes_etal_2013}. Hence, stomatal traits like density, size, and ratio of upper to lower stomata have strong effects on carbon assimilation and water-use efficiency.

% The proportion of stomata found on the upper surface also tends to increase during domestication, even as the total stomatal density stays constant \citep{Milla_etal_2013}. Amphistomy increases CO$_2$ diffusion within the leaf by opening up a second parallel pathway in the intercellular airspace for diffusion from substomatal cavities to mesophyll cell walls. However, stomata on the upper surface in particular may be costly. For example, upper stomata increase the susceptibility to rust pathogens in \textit{Populus} \citep{Mckown_etal_2014}. Amphistomy may also cause the palisade mesophyll to dry out under strong vapor pressure deficits \citep{Buckley_etal_2015}. \cite{Muir_2015} reviewed the literature on other possible fitness costs. 

%--------------------------------------------------
% Methods
%--------------------------------------------------

\section*{METHODS}

Data and annotated source code to generate this manuscript are available on GitHub (https://github.com/cdmuir/britstom) and Dryad \citep{Muir_dryad}.

\subsection*{Data on stomatal ratio, light habitat, growth form, and phylogenetic relationships}

I obtained data on ab- and adaxial stomatal density on 395 species from British Ecological Flora \citep{Salisbury_1927, Fitter_Peat_1994a, BEF}. Following recent comparative analyses \citep[e.g.][]{Bartelheimer_Poschlod_2016, Salguero-Gomez_etal_2016}, I used Ellenberg light indicator values \citep{Ellenberg_1974} and Raunki\ae r life form \citep{Raunkiaer_1934} as measures of light habitat and growth form, respectively. Hence, I am assuming that the species' light habitat is closely related to the type of habitat (open versus closed) where that species is found. Both attributes have been recently updated by taxonomic experts of the British flora (PLANTATT, \cite{Hill_etal_2004}). Ellenberg light indicator values are hereafter abbreviated \el. I used a dated molecular phylogeny of the British flora \citep{Lim_etal_2014} available from TreeBASE (http://treebase.org/; accession number 15105). 14 species (3.5\%) in the dataset were not present in the phylogeny. For 8 of these species, I used the position a congeneric species as a proxy for the focal species. When multiple congeneric species were present, I consulted the phylogenetic literature to identify the most closely related proxy species \citep{Scheen_etal_2004, Salmaki_etal_2013}. For the remaining 6 missing species, I positioned them in the tree based on phylogenetic relationships to other genera or families present in the tree \citep{Fior_etal_2006}. Because many phylogenetic comparative methods do not allow polytomies, zero-length branches, and non-ultrametric trees, I made several small adjustments to the tree. I resolved polytomies randomly using the `multi2di' function in \pkg{phytools} version 0.5-64 \citep{Revell_2012}. I added 0.02 my to all zero-length branches, as this was approximately the length of the shortest nonzero branch length in the tree. After these changes, I slightly altered terminal branch lengths to make the tree precisely ultrametric.

I excluded data on hyrdrophytes (14 species) because many of these species are hyperstomatous (Fig. \ref{fig:violin}) due to the fact that leaves may rest on the water's surface, selecting for stomata to be present on the upper surface only. I also excluded C$_4$ (3 species) and CAM (2 species) plants. I limited this investigation to angiosperms because only 4 non-angiosperms had stomata data. The final dataset contained 372 species. The R code accompanying this paper documents these decisions with citations to the relevant literature.

Following \cite{Muir_2015}, I calculated stomatal ratio in two different ways depending on what was most appropriate for the question: 

\begin{equation} \label{eq:SRpropAd} 
  \mathrm{SR_{propAd}} = \frac{\mathrm{SD_{ad}}}{\mathrm{SD_{total}}}
\end{equation}

\begin{equation} \label{eq:SReven1} 
  \mathrm{SR_{even}} = \frac{\mathrm{min}\{\mathrm{SD_{ab}}, \mathrm{SD_{ad}}\}}{\mathrm{max}\{\mathrm{SD_{ab}}, \mathrm{SD_{ad}}\}}
\end{equation}

$\mathrm{SD_{ab}}$ and $\mathrm{SD_{ad}}$ are the stomatal densities on abaxial or adaxial surface, respectively. $\mathrm{SD_{total}} = \mathrm{SD_{ab}} + \mathrm{SD_{ad}}$. $\mathrm{SR_{propAd}}$ is the proportion of stomata density on the adaxial surface, which is useful for discriminating among hypostomatous ($\mathrm{SR_{propAd}} = 0$), amphistomatous (0 < $\mathrm{SR_{propAd}} < 1$), and hyperstomatous species ($\mathrm{SR_\mathrm{propAd}} = 1$). $\mathrm{SR_\mathrm{even}}$ indicates how evenly stomatal densities are distributed across both leaf surfaces. This expression is useful because several hypotheses are based on the fact that a more even distribution should optimize leaf CO$_2$ diffusion.

% NOTES

%  papers on Ellenberg light value:
%  Bartelheimer and Poschlod 2015
%  Diekmann, 2003

\subsection*{Testing for an association between open habitat and growth form}

I tested whether Raunki\ae r life form was associated \el~values among British angiosperms using ANOVA with Type-2 sum of squares. I did not use phylogenetic ANOVA for this test because there was no phylogenetic signal in the regression fit using \pkg{phylolm} version 2.5 \citep{Ho_Ane_2014}. See the R code accompanying this paper for further detail. I predicted that species with faster life histories, especially therophytes (annuals), would have greater \el~than species with slower life histories, especially phanerophytes, which are mostly long-lived trees. 

\subsection*{Open habitat, growth form, and stomatal ratio}

I compared phylogenetic linear models to test whether Raunki\ae r life form, \el, or interactions between them predicted $\textrm{SR}_\textrm{even}$. I used $\textrm{SR}_\textrm{even}$ rather than $\textrm{SR}_\textrm{propAd}$ as the response variable because the hypothesis is that faster life history and/or high light favor more even stomatal densities on each surface. I fit models using \pkg{phylolm} and extracted Akaike Information Criteria (AIC). For these and subsequent analyses, I assumed an Ornstein-Uhlenbeck process model for the residuals with the root character state integrated over the stationary distribution. I used a 10,000 parametric bootstrap samples of the full model (including main effects and interactions) to calculate parameter confidence intervals \citep{Boettiger_etal_2012}. Likewise, to determine whether the interaction between Raunki\ae r life form and \el~was statistically significant, I used a parametric bootstrap to generate the null distribution of $\Delta$AIC values ($\Delta$AIC is the difference in AIC between competing models). Specifically, I sampled 1000 random datasets from the estimated model with main effects of Raunki\ae r life form and \el~but no interaction. I fit these simulated datasets to models with and without interactions and  calculated $\Delta$AIC. The statistical significance of the observed $\Delta$AIC is the proportion of simulated $\Delta$AIC greater than the observed.

\subsection*{Does ab- or adaxial stomatal density contribute more to stomatal ratio evolution?}

I used two complementary phylogenetic methods to assess the relative contribution of ab- versus adaxial stomatal density to light-mediated stomatal ratio evolution. The contribution of each can be formalized using standard variance decomposition methods as derived below. Because stomatal density is highly skewed, I log-transformed values for normality:
 
\begin{equation} \label{eq:SReven2} 
  \mathrm{SR_{even}} = \frac{\mathrm{SD_{ad}}}{\mathrm{SD_{ab}}}
\end{equation}

\begin{equation} \label{eq:logSReven} 
  \mathrm{log(SR_{even})} = \mathrm{log(SD_{ad})} - \mathrm{log(SD_{ad})}
\end{equation}

\begin{equation} \label{eq:SReven3} 
  \mathrm{sr_{even}} = \mathrm{sd_{ad}} - \mathrm{sd_{ad}}
\end{equation}

Lowercase variables ($\mathrm{sr}$, $\mathrm{sd}$) indicate log-transformed values. Because some species had zero adaxial stomata, I added one to all values prior to log-transformation. For simplicity, I have defined $\mathrm{SR_{even}}$ here as the ratio of ad- to abaxial stomatal density because in most cases adaxial stomatal density is lower than abaxial (see Eq.~\ref{eq:SReven1}). The variance in $\mathrm{sr_{even}}$ can be decomposed into contributions of $\mathrm{sd_{ad}}$, $\mathrm{sd_{ab}}$, and their covariance:

\begin{equation} \label{eq:varDecomp}
	\mathrm{Var(sr_{even})} = \mathrm{Var(sd_{ad})} + \mathrm{Var(sd_{ad})} - 2 \mathrm{Cov(sd_{ad}, sd_{ab})}
\end{equation}

I estimated the phylogenetic covariance matrix between \el, $\mathrm{sd_{ab}}$, and $\mathrm{sd_{ad}}$ using a multivariate Ornstein-Uhlenbeck model fit in \pkg{Rphylopars} version 0.2.9 \citep{Goolsby_etal_2016, Goolsby_etal_2017}. From the covariance matrix, I estimated the contribution of abaxial density, adaxial density, and their covariance as:

\begin{equation} \label{eq:contribution}
	\frac{\mathrm{Var(sd_{ad})}}{\mathrm{Var(sr_{even})}}, \frac{\mathrm{Var(sd_{ab})}}{\mathrm{Var(sr_{even})}},~\textrm{and}~\frac{\mathrm{Cov(sd_{ad}, sd_{ab})}}{\mathrm{Var(sr_{even})}},
\end{equation}

respectively. Note that when ab- and adaxial densities positively covary, the contribution will be negative because this reduces the variance in stomatal ratio.

I was interested in whether light-mediated evolution of stomatal ratio acted mostly by increasing adaxial stomatal density while maintaining abaxial density, or keeping total stomatal density the same, but shifting a greater proportion to the adaxial surface. The first scenario predicts that the phylogenetic regression of \el~ on $\mathrm{sd_{ad}}$ is stronger than that for $\mathrm{sd_{ab}}$. The second scenario predicts that \el~acts similarly on both and that there is a negative covariance $\mathrm{Cov(sd_{ad}, sd_{ab}) < 0}$. I tested these competing predictions by fitting a simple phylogenetic structural equation model (SEM). The model uses the phylogenetic covariance matrix to simultaneously estimate regressions of \el~on $\mathrm{sd_{ad}}$ and $\mathrm{sd_{ab}}$ while allowing covariance between them (i.e. estimating $\mathrm{Cov(sd_{ad}, sd_{ab})}$). To fit the SEM, I used the R package \pkg{lavaan} version 0.5-23.1097 \citep{Rosseel_2012}. I tested whether parameter estimates were significantly different than zero using $z$-scores.

%--------------------------------------------------
% Results
%--------------------------------------------------

\section*{RESULTS}

\subsection*{Light tolerance varies among Raunki\ae r life forms}

Ellenberg light indicator values (\el) differed significantly among life forms (Fig.~\ref{fig:lf-light};ANOVA - $F_{4, 367}$ = 18.3, $P$ = 1.1 $\times10^{-13}$). Therophytes (annuals), hemicryptophytes (perennial herbs with buds near the soil surface), and chamaephytes (subshrubs) had greater \el~than phanerophytes (large woody plants) and geophytes (perennial herbs with storage organs) (Fig.~\ref{fig:lf-light}).

\subsection*{Interactions between light and Raunki\ae r life form determine stomatal ratio}

Overall, $\mathrm{SR_{even}}$ increased with \el, but there was a significant interaction between Raunki\ae r life form and \el~(Fig.~\ref{fig:SRmultReg}). Both life form and \el~significantly increased model fit, though \el~had a markedly larger effect on model AIC (Table~\ref{table:srAIC}). The significant interaction is caused by different slopes between life forms. Among life forms with the overall greatest \el~(therophytes, hemicryptophytes, and chamaephytes, see Fig.~\ref{fig:lf-light}), there was a strong positive relationship between \el~and $\mathrm{SR_{even}}$. Parametrically bootstrapped 95\% confidence intervals did not overlap zero (Fig.~\ref{fig:SRmultReg}). The slope was weakly positive or not significantly different from zero in the most shade-adapted life forms (geophytes and phanerophytes), albeit the patterns were distinct in these groups. There were both hypostomatous ($\mathrm{SR_{even}} \approx 0$) and amphistomatous ($\mathrm{SR_{even}} \approx 1$) geophytes, but these were distibuted across \el s. In contrast, phanerophytes were nearly always hypostomatous regardless of \el. Allowing slopes to vary across life form signicantly increased model fit (lower AIC, Table~\ref{table:srAIC}).

\subsection*{Adaxial stomatal density contributes most of the variation in stomatal ratio}

Adaxial (`upper') stomatal density contributed most to the evolutionary variation in stomatal ratio. The contributions of adaxial density, abaxial density, and their covariance are 1.14, 0.38, and -0.53, respectively. Recall that values can be greater than one for adaxial stomatal density and negative for the covariance when the latter value is positive. This implies that evolutionary variation in adaxial stomatal density is greater than that for stomatal ratio due to positive covariance between ab- and adaxial stomatal density.

Similarly, the phylogenetic SEM showed that changes in stomatal ratio associated with \el~can be attributed mostly to evolution of adaxial stomatal density (Fig.~\ref{fig:SD-light}). Both $\mathrm{sd_{ad}}$ and $\mathrm{sd_{ab}}$ increased with \el~($P =$ 6.1 $\times10^{-7}$ and 2.9 $\times10^{-5}$, respectively). However, the regression of \el~on $\mathrm{sd_{ad}}$ was 2.1$\times$ that of \el~on $\mathrm{sd_{ab}}$ (0.21 versus 0.1). Because stomatal densities were natural log-transformed, this implies an increase in \el~by one leads to a 1.23-fold change in adaxial stomatal density versus a 1.1-fold change in abaxial stomatal density. The SEM also showed a significant positive covariance between stomatal densities on each surface ($P = $1.7 $\times10^{-11}$). These results together imply that total stomatal density increases with \el, but the response is mediated mostly by adaxial stomatal density.

% NOTES FOR CALCULATING fold-chage
% log(sd1) = b1 * 3
% SD1 = exp(b1) ^ 3
% SD1 = exp(b1) ^ 4
% change from 3 to 4 is exp(b1)
% for SD2, change from 3 to 4 is exp(b2)

%--------------------------------------------------
% Discussion
%--------------------------------------------------

\section*{DISCUSSION}

The ratio of stomatal densities on the abaxial (`lower') to that of the adaxial (`upper') surface varies greatly across plant species, but the adaptive significance is not clear. Comparative studies correlating stomatal ratio to ecological factors can distinguish among competing hypotheses and reveal critical experiments for future work. Previous comparative studies suggested that high light and herbaceous growth form favor amphistomy \citep{Mott_etal_1982, Jordan_etal_2014, Muir_2015, Bucher_etal_2017}, particularly in the British flora \citep{Salisbury_1927, Peat_Fitter_1994b}. However, none of these studies have accounted for the fact thats light and growth form are often confounded -- open, high light habitats are necessarily dominated by herbs -- or the fact that species are not independent because of shared evolutionaey hisory. Here, I reanalyzed data on stomata, light tolerance, and growth form in British angiosperms using phylogenetic comparative methods. As expected, species' light tolerance (Ellenberg light indicator or \el) is confounded with growth form (Raunki\ae r life form; Fig.~\ref{fig:lf-light}). Nevertheless, both \el~and Raunki\ae r life form affect stomatal ratio, but these factors also interact; the influence of \el~on stomatal ratio varies across forms. These novel findings provide further evidence that variation in stomatal ratio is adaptive and have important implications for interpreting changes in stomatal ratio through the paleo record \citep{Jordan_etal_2014} and during domestication \citep{Milla_etal_2013}.

\subsection*{Adaptive significance of amphistomy}

Previously, associations between light, growth form, and stomatal ratio have been interpreted in isolation as indicating that either high light and/or herbaceous growth form favors amphistomy. In British angiosperms, both factors are important, though statistical analyses suggest that light may be a stronger determinant than growth form (Table~\ref{table:srAIC}). Unlike previous studies, I found a significant interaction between light and growth form among British angiosperms, which suggests that amphistomy may only be strongly favored when CO$_2$ strongly limits photosynthesies \textit{and} photosynthesis strongly limits fitness. The ideal way to test this would be to measure selection on stomatal ratio in a species that varied quantitatively in both stomatal ratio and life history (e.g. containing both annual and perennial forms). I predict that amphistomy will be favored much more strongly in the annual form grown under high light compared to an annual under low light or a perennial in high light. Similar experiments could also be performed to test if and when light-mediated plasticity in stomatal ratio is adaptive \citep{Gay_Hurd_1975, Mott_Michaelson_1991}.

The prevalence of amphistomatous species in high light habitats supports the hypothesis that amphistomy is an adaptation to maximize photosynthetic rates by increasing CO$_2$ diffusion \citep{Jones_1985}. It is also evidence against the hypothesis that the principle fitness cost of amphistomy is water loss \citep{Darwin_1886, Foster_Smith_1986} or dehydration of pallisade mesophyll \citep{Buckley_etal_2015}. Since evaporative demand increases under high insolation, under these hypotheses we would expect plants in high light to be hypostomatous. Because stomatal conductances on each surface can be regulated independently in response to the environment \citep{Darwin_1898, Pospisilova_Solarova_1980, Smith_1981, Reich_1984, Mott_Oleary_1984}, amphistomatous leaves likely cope with these stresses by rapidly closing adaxial stomata when water supply cannot match evaporative demands. Instead, patterns in the British flora are at least consistent with the idea that adaxial stomata increase susceptibility to foliar pathogens \citep{Gutschick_1984b, Mckown_etal_2014}. The cost of adaxial stomata may be greater in the shade because greater leaf wetness and lower ultraviolet light provide a more suitable microclimate for many foliar pathogens.

\subsection*{Amphistomy as a proxy for open habitat}

These observations from the British flora strongly support the hypothesis that amphistomy can be used a proxy for open habitat in paleoenvironment reconstruction \citep{Carpenter_1994, Jordan_etal_2014, Carpenter_etal_2015}, but also point out previously unknown subtleties. These previous studies based their conclusions on data from Proteaceae, in which there is little quantitative variation in stomatal ratio; species are either completely hypostomatous ($\mathrm{SR_{propAd}} \approx 0$) or completely amphistomatous ($\mathrm{SR_{propAd}} \approx 0.5$). Stomatal ratio in British angiosperms is also bimodal \citep{Peat_Fitter_1994b}, but across many families there is also quantiative variation. Importantly, this means that quantitative variation in stomatal ratio may provide a more precise, quantiative indicator of vegetation type, rather than simply `open' or `closed'. A quantitative relationship between \el~and stomatal ratio has already been shown for herbaceous perennials \citep{Bucher_etal_2017}, but we now know that it holds among annuals (therophytes), subshrubs (chamaephytes), and, to a lesser extent, geophytes as well (Fig.~\ref{fig:SRmultReg}). 

The nonsignificant relationship between \el~and stomatal ratio in geophytes and phanerophytes suggests that in some cases amphistomy may not reliably indicate open habitat without further information. For example, perhaps amphistomatous geophytes from partially shaded habitats are spring ephemerals, so they  experience high light during their growth phase, but this has not been tested. Likewise, phanerophytes (most tall trees) are almost always hypostomatous (see also \cite{Muir_2015}). Most British phanerophytes are tall, hypostomatous trees, but the exceptions are telling. For example, the most amphistomatous phanerophyte in this dataset is \textit{Brassica oleracea}, a short-statured biennial that has more in common physiologically with hemicryptophytes than other phanerophytes. The other amphistomatous phanerophytes in this data set (\textit{Populus nigra} and \textit{Lavatera arborea}) are fast-growing pioneer species.

Finally, phylogenetic information should improve inferences about paleoclimates because there is appreciable phylogenetic signal in stomatal ratio. The phylogenetic half-life of stomatal ratio evolution, after accounting for \el~and Raunki\ae r life form, is $\mathrm{log}(2) / \alpha = 1.5$ my (see Table~\ref{table:srAIC} for maximum likelihood estimates of $\alpha$). This lag time may indicate that evolving to the `optimum' is constrained by the shape of the fitness landscape \citep{Muir_2015} or that other unmeasured factors which affect stomatal ratio have some phylogenetic signal. Regardless of the mechanism, this fact means that researchers may be able to use data from closely related species to improve paleoenvironment reconstruction.

% phylo half-life estimate from Muir 2015 is 22my

\subsection*{Why does adaxial stomatal density control stomatal ratio?}

Variation in stomatal ratio is determined primarily by evolution of adaxial stomatal density and is coordinated with increases in total leaf stomatal density summed across both surfaces. Phylogenetic analyses show that changes in stomatal ratio and total stomatal density, especially in response to \el, are predominantly mediated by changes in adaxial stomatal density. This highly nonrandom pattern among British angiosperms mirrors evolutionary changes wrought by domesication \citep{Milla_etal_2013}; crops species tend to have higher adaxial stomatal density than their wild relatives. Note here that I am referring only to evolutionary variation in stomatal ratio among species; different processes may mediate within species variation or plastic responses.

There are at least two hypotheses that could explain why adaxial stomatal density is the most responsive. The first I refer to as the `real estate' hypothesis. In hypostomatous plants, the lower surface is already crowded with stomata, and hence plants must increase the real estate available for stomata by develping them on the upper surface whenever there is selection for greater stomatal density. When stomata are packed too densely on one surface, stomatal interference limits their functioning and hence may create a strong selective pressure for amphistomy \citep{Parlange_Waggoner_1970, Dow_etal_2014a}. 

I refer to the second hypothesis as the `coordination' hypothesis. In this scenario, ecological conditions such as high light select for both increased total stomatal density and for amphistomy because these traits work well in coordination with one another. For example, if stomatal density were very high on a hypostomatous plant, then CO$_2$ would be more strongly limited by the mesophyll. Adding a second parallel pathway for diffusion by developing stomata on both surfaces would restore a more optimal balance between stomatal and mesophyll limitations. Conversely, there would be little benefit to amphistomy when total stomatal density is low because CO$_2$ diffusion is strongly limited by stomatal resistance, and therefore photosynthetic rate is not sensitive to changes in mesophyll diffusion mediated by stomatal ratio.


% alternatives are different benefits and costs
% leaf orientation \citep{Smith_etal_1998}, stomatal interference , leaf hydraulic conductance (Sack, Buckley, de Boer?), change in stomatal ratio may be important to CO2 in amphi species (Vaccinium myrttillus, \citep{Woodward_Bazzaz_1988})

\subsection*{Conclusions - finish when analysis is complete}








% From previous paper:
% Adaptive evolution may be constrained if traits cannot evolve independently. In particular, if traits share developmental pathways, then they may be unable to respond differentially to selection. In this study, we examined whether stomata on the abaxial (lower) and adaxial (upper) surfaces can evolve independently. We adduce two new lines of evidence which suggest that stomatal function on each surface can readily respond to selection. First, species possess heritable variation that allows partially independent evolution of stomatal densities in response to selection; every study reviewed found loci which alter stomatal density on one surface but not the other. Second, the anatomical trait most closely connected to stomatal conductance, stomatal pore index \citep{Sack_etal_2003}, evolves independently on ab- and adaxial surfaces among wild tomato species. Together, these new lines of evidence demonstrate that natural selection on stomatal arrangement is not strongly constrained by development, although we lacked statistical power to detect weak constraint. It is therefore likely that variation in how stomatal conductance is partitioned between leaf surfaces is due to adaptive rather than nonadaptive forces.

% Indeed, much recent evidence indicates that selection finely tunes the ratio of stomata on the upper and lower leaf surface, although the adaptive significance of variation in stomatal ratio is unresolved. Stomatal ratio affects leaf function, increasing CO$_2$ diffusion \citep{Parkhurst_1978, Parkhurst_Mott_1990, Gutschick_1984b, Parkhurst_1994} and hydraulic conductance outside the xylem \citep{Buckley_etal_2015}. As predicted, amphistomy seems to be more common in circumstances when efficient CO$_2$ supply is important, such as high irradiance \citep{Mott_etal_1982, Gibson_1996, Smith_etal_1997, Jordan_etal_2014}, thick leaves \citep{Parkhurst_1978, Muir_etal_2014b}, herbaceous growth form \citep{Salisbury_1927, Muir_2015}, and domestication \citep{Milla_etal_2013}. Despite potential benefits of amphistomy, most plant species are hypostomatous, implying a fitness cost of upper stomata, such as increased infection by foliar pathogens \citep{Gutschick_1984b, Mckown_etal_2014}. For example, `upside-down' (resupinate) leaves with the abaxial surface facing upward have re-evolved hypostomy \citep{Lyshede_2002}, strongly implying a cost of upward facing stomata. 

% To optimally balance fitness costs and benefits, natural selection must be able to change stomatal traits on one surface independently of the other. The present study shows that this is likely true and strikingly consistent on micro- and macroevolutionary timescales. Among \textit{Populus trichocarpa} populations and \textit{Solanum} species, the ratio of adaxial to abaxial $\mathrm{SPI}$ ($\mathrm{SPI}$ ratio) evolves mostly by changes in stomatal density rather than guard cell size. Within \textit{Populus}, populations are more amphistomatous at Northern latitudes with shorter growing seasons that may select for faster carbon assimilation \citep{Mckown_etal_2014, Kaluthota_etal_2015}. Latitudinal variation \textit{Populus trichocarpa} is due mostly to adaptive variation in adaxial stomatal density \citep{Mckown_etal_2014, Porth_etal_2015}. Stomatal density rather than size may have responded more readily to selection because there is no genetic covariance between ab- and adaxial stomatal density, permitting independent evolution \citep{Porth_etal_2015}. In contrast ab- and adaxial guard cell length positively covary, likely constraining evolution. Similarly, we found that over macroevolutionary timescales most of the variation in $\mathrm{SPI}$ among wild tomato species is due to changes in adaxial stomatal density rather than size. Indeed, stomatal density on each surface evolved independently, whereas guard cell lengths positively covaried (Table~\ref{table:Table2}). Adaptive evolution will likely take advantage of traits that evolve independently because this minimizes antagonistic pleiotropy. In a previous study, we found that loci affecting adaxial stomatal density were likely fixed by selection, but we did not measure stomatal size \citep{Muir_etal_2014a}. Overall, patterns within and between species indicate that selection on $\mathrm{SPI}$ ratio leads to greater change in stomatal densities rather sizes on each surface. Based on the analysis here, we conclude that changing stomatal density on one surface incurs less cost than changing size because the former is less constrained by shared developmental pathways.

% However, simulations show that model identification (e.g. Brownian motion versus Ornstein-Uhlenbeck) is usually correct, even when sample sizes are moderate \citep{Cressler_etal_2015, Ho_Ane_2014}. 


\clearpage

%--------------------------------------------------
% Literature Cited
%--------------------------------------------------

\bibliography{refs}
\bibliographystyle{evolution}

\clearpage

%--------------------------------------------------
% Tables
%--------------------------------------------------

\begin{table}[ht]
 \setstretch{\stretchy}
  \topcaption{Interaction beween species' Ellenberg light indicator value (\el) and Raunki\ae r lifeform shape stomatal ratio ($\mathrm{SR_{even}}$). I compared phylogenetic linear models using the Akaike Information Criterion (AIC), where AIC = $2k - 2 \mathrm{log}(\mathcal{L})$. $k$ is the number of model parameters and $\mathcal{L}$ is the model likelihood. Given a set of candidate models, the difference in AIC between a model and the lowest AIC ($\Delta$AIC) indicates the relative fit of competing models. The correlation coefficient $r^2$ is another indicator of model fit. $\alpha$ and $\sigma^2$ are the return rate and diffusion parameters of the Ornstein-Uhlenbeck model of trait evolution. }
  \begin{center}
  \begin{tabular}{@{} l c c c c c c c @{}}
  \toprule
  Model: $\mathrm{SR_{even}} \sim$ &  $\alpha$ & $\sigma^2$ &  $r^2$ & $k$ &  log($\mathcal{L}$) &  AIC &  $\Delta$AIC \\
  \midrule  
    \el~$\times$ lifeform
  & 0.46
  & 0.068
  & 0.34
  & 12
  & -33.3
  & 90.6
  & 0 \\
    \el~+ lifeform
  & 0.47
  & 0.072
  & 0.32
  & 8
  & -40.3
  & 96.5
  & 6 \\
    \el
  & 0.64
  & 0.108
  & 0.26
  & 4
  & -59.3
  & 126.6
  & 36.1 \\
    lifeform
  & 0.34
  & 0.067
  & 0.15
  & 7
  & -79.2
  & 172.4
  & 81.8 \\
    1
  & 0.29
  & 0.067
  & 0
  & 3
  & -107.5
  & 221
  & 130.5 \\
	\bottomrule
	\end{tabular}
	\label{table:srAIC}
\end{center}
\end{table}

%--------------------------------------------------
% Figures
%--------------------------------------------------

\begin{figure}[ht]
\centerline{\includegraphics[width=0.5\textwidth]{figures/figure_lf-light.pdf}}
\caption{Lifeforms have different tolerances for sun and shade among British angiosperms. Each panel is the distribution of Ellenberg light indicator values on an integer scale of 1-9 for different Raunki\ae r life forms. Height of the bars indicate the raw proportion of species in each bin for that lifeform. The sample size for each lifeform is listed next in parentheses. The mean (open circle) and 95\% confidence intervals (black line) around the mean Ellenberg light indicator value for each lifeform based on phylogenetic regression are above the histogram.} 
\label{fig:lf-light}
\end{figure}

\begin{figure}[ht]
\centerline{\includegraphics[width=0.5\textwidth]{figures/figure_SRmultReg.pdf}}
\caption{The effect of light on stomatal ratio depends on Raunki\ae r life form. Greater Ellenberg light indicator values (\el) are associated with greater stomatal ratio ($\mathrm{SR_{even}}$) in therophytes, hemicryptophytes, and chamaephytes but not geophytes and phanerophytes. The maximum likelihood slope from phylogenetic regression is given with statistical significance based on 1000 parametric bootstrap samples. Numbers in parentheses next to Raunki\ae r life form are the sample sizes in the final dataset. Estimated slopes (solid line) and 95\% bootstrapped confidence intervals (gray polygon between dashed lines) are plotted against raw data. Points have been jittered for visial clarity.} 
\label{fig:SRmultReg}
\end{figure}

\begin{figure}[ht]
\centerline{\includegraphics[width=0.5\textwidth]{figures/figure_SD-light.pdf}}
\caption{Light-mediated evolution of stomatal ratio is mostly driven by increased adaxial (`upper') stomatal density (Panel A), whereas abaxial (`lower') stomatal density (Panel B) is similar across Ellenberg light indicator values (\el~$x$-axis). The violin plot shows stomatal density ($y$-axis, log-scale) as a function of \el. The width of the grey polygons indicates the density of data. Length of grey polygon indicate the range of the data; the point indicates the median; the thick lines indicate the 0.25 and 0.75 quantiles. Points outside the polygons are statistical outliers.} 
\label{fig:SD-light}
\end{figure}

\clearpage

%--------------------------------------------------
% Supporting Information
%--------------------------------------------------

\section*{Supporting Information}

% Modify and restart table/figure numbering for appendixes
\renewcommand\thefigure{S\arabic{figure}}    
\renewcommand\thetable{S\arabic{table}}    
\renewcommand\theequation{S\arabic{equation}}    
\setcounter{table}{0}    
\setcounter{equation}{0}
\setcounter{figure}{0}

%--------------------------------------------------
% Supporting Figures
%--------------------------------------------------

\begin{figure}[ht]
\centerline{\includegraphics{figures/figureS_violin.pdf}}
\caption{Most hydrophytes are hyperstomatous, having most stomata on the adaxial (`upper') surface (high $\mathrm{SD_{propAd}}$). The violin plot shows stomatal ratio as a function of Raunki\ae r lifeform. The width of the grey polygons indicates the density of data. Length of grey polygon indicate the range of the data; the point indicates the median; the thick lines indicate the 0.25 and 0.75 quantiles. Sample sizes per lifeform in the dataset are given above the upper plot margin. $\mathrm{SD_{ad}}$ and $\mathrm{SD_{total}}$ stand for the stomatal density on the adaxial surface and the total leaf surface (adaxial plus abaxial), respectively.}
\label{fig:violin}
\end{figure}

\end{document}
